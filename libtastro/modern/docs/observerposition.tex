\mychapter{Position of the observer}

Calendrical calculations are supposedly made by humans above the sea, not at the geocenter. So the observed longitude of a celestial body 

$R_o$ is the distance of the observer from the geocenter ($R_E+altitude$)
$D_{E-M}$ is the Earth-Moon distance
$\Theta$ is the resulting error

The error will be maximized by an observation of the moon when it's $90°$ from vernal equinox and at its perigee (about $D_{E-Mmin}\approx360000km$) made by someone on the top of the Everest ($R_o=R_E+8,848km$) and considering the Everest is on the Ecliptic. In this case we would have:

$$\Theta_{max} = atan(R_o/D_{E-Mmin}) = 3655as$$

% atan(6379.85/360000) = .01771995065419387521rad = 3655s

Which is an error to be avoided by all means!

If we take an observation at sea level, with a $30°$ angle and an average distance (Sun-Earth distance for inner planets and Sun-planet distance for outer ones), we can compute the following table:

\begin{center}
\sisetup{table-figures-exponent = 1}
\begin{tabular}{|l|S|S|}
\hline
\textbf{Celestial body} & \multicolumn{1}{c|}{\textbf{$\Theta_{max}$}} & \multicolumn{1}{c|}{\textbf{$\Theta$}}  \\\hline
Sun & 9 & 4\\\hline % min: 147Mkm, mean: 150km
Moon & 3655 & 1706\\\hline % min: 360kkm, mean: 385kkm
Mercury & 17 & 7\\\hline %  min: 77 Mkm, mean: 100Mkm
Mars & 24 & 3\\\hline % min: 54.6Mm, mean: 225Mkm
Venus & 35 & 4\\\hline % min: 38Mkm, mean: 150Mkm
Saturn & 1 & 453\\\hline % min  1200Mkm, mean: 1450Mkm
Jupiter & 2e-3 & 8e-4 \\\hline % min: 629kMkm, mean: 770kMkm
\end{tabular}
\end{center}

How to take it into consideration?

Thoughts: position of the observer actually changes the phase of the moon!

when to take it into account? not for mean

position of the observer in light travel time consideration

phase of the moon and light travel time (!) (to go in another section)
