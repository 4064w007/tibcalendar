\documentclass[%
a4paper,% paper size
pagesize,%
12pt,% text size
parskip=off,% inter-paragraph space, automatically sets paragraph indentation
bibliography=totoc,% bibliography in the table or contents
numbers=noenddot,%
DIV=12,% margin settings
twoside=semi,% right and left difference, but same margins
headings=small,% small headings
]{scrbook}

\usepackage{eroux}

\title{Astronomical calculations in lunisolar calendars}
\subtitle{Overview of accuracy and optimization possibility}
\author{Élie Roux \href{mailto:elie.roux@telecom-bretagne.eu}{<elie.roux@telecom-bretagne.eu>}} 
\date{\today} 

\pagestyle{scrheadings}
\ifoot[]{} 
\cehead[]{\textsc{\rightmark}}
\cohead[]{\textsc{Astronomical calculations in lunisolar calendars}}
% et enfin le numéro de page dans le pied de page extérieur
\ofoot[\oldstylenums\thepage]{\oldstylenums\thepage}
\cfoot[]{}
\ohead[]{}
\ihead[]{}

\let\mychapter\chapter
\let\mysection\section
\let\mysubsection\subsection
\let\mysubsubsection\subsubsection

\def\TODO{}

\begin{document}

\maketitle

\tableofcontents
%\cleardoubleevenemptypage
%\clearpage
\newpage

\mychapter{Introduction}

\mysection{What it this document?}

This article aims at giving an overview of the main inaccuracies encounterable in astronomical calculations when used in calendrical calculations.

\mysection{Genesis}

Working on Tibetan calendar calculations, I realized that there was no documentation on how to use the main astronomical calculations (such as VSOP87) to get the data necessary to build a calendar. These include the different approximations we would naturally make with the result, the induced discrepancy of which I have never seen documented.

I guess many people did these calculations, such as the authors of the main astronomical libraries, but no documentation is available on the global accuracy of their calculations nor on why they would take into account a phenomenon or not.

Instead of redoing these calculations myself, I decided to compute them fully and document them, in order:
\begin{itemize}
\item to make the accuracy gain of some considerations documented
\item to analyze and document the precision of the calculations used in my astronomical library\TODO
\end{itemize}

This article is made from the point of view of calendrical calculations but will be meaningful with any calculation from the point of view of a human observer on the Earth.

This article will not give a very precise average error for all phenomena, but:
\begin{itemize}
\item the maximum error
\item an error computed with what seems average values
\item a conclusion on the necessity to take the phenomenon into account according to the desired precision
\end{itemize}



\mysection{Conventions used in this article}

\mysubsection{Assertions}

In this article, we will consider that latest JPL ephemeris \TODO are fully accurate. 

We consider that all computer calculations are approximated as defined by IEEE 754/784, so calculations are supposed to happen in C99 or Fortran 2003. If you use a non-prehistoric processor and compiler, it should be the case though.

\mysubsection{Notation and vocabulary}

\mysubsubsection{Trueness and precision}

This is quite important here to differenciate trueness and precision, both being part of accuracy. These terms are officially defined in \cite{VIM} as closeness of agreement between:

\begin{description}
\item[trueness] the average of an infinite number of replicate measured quantity values and a reference quantity value
\item[precision] indications or measured quantity values obtained by replicate measurements on the same or similar objects under specified conditions
\item[accuracy] a measured quantity value and a true quantity value of a measurand
\end{description}

It makes sense to translate them to astronomical calculations, taking a calculation as a \emph{measure}. Figure~\ref{ATP} is useful to get a better understanding.

\begin{figure}[h]
\centering
\def\svgwidth{10cm}
\input{ATP.pdf_tex}
%\caption{Accuracy, Precision and Trueness of calculations\\This image comes from Wikipedia and is under the same license as this document\TODO}
\label{ATP}
\end{figure}

\mysubsection{Astronomical values}

\mysubsubsection{Orbital caracteristics}

For future calculations, we'll need a set of orbital values which we'll present and justify here. The values we'll use for $a$ and $e$, presented in Table \cite{table:planetorbitalvalues}, come from \cite{NASA-factsheet}. % (TODO: http://nssdc.gsfc.nasa.gov/planetary/planetfact.html)

\begin{table}
\centering
\begin{tabular}{|l|S[table-format=1.3,table-figures-exponent=1]|S[table-format=1.2,table-figures-exponent=1]|}
\hline
\multicolumn{1}{|c|}{\textbf{Celestial body}} & \multicolumn{1}{c|}{\textbf{$a$} (km)} & \multicolumn{1}{c|}{\textbf{$e$}} \\\hline
Sun (Earth) & 1.496e8 & 1.67e-2\\\hline
Moon & 3.84e5 & 5.49e-2\\\hline
Mercury & 5.79e7 & 2.056e-1\\\hline
Mars & 2.279e8 & 9.35e-2\\\hline
Venus & 1.082e8 & 6.77e-3\\\hline
Saturn & 1.433e9 & 5.65e-2\\\hline
Jupiter & 7.786e8 & 4.89e-2 \\\hline
\end{tabular}
\caption{Planet caracteristics used in the calculations}
\label{table:planetorbitalvalues}
\end{table}

Then we need the maximum speed of celestial bodies. The first estimation of it (which we'll use) is obtained by the Kepler laws: $$\sqrt{2GM}\times \sqrt{\frac{1}{r}-\frac{1}{2a}}$$. We use $G=6.67384\times 10^{2} km^3kg^{-1}s^{-2}$\footnote{As recommended by \cite{CODATA}}, $GM_{Sun}=1.3271244\times 10^{11} km^3s^{-2}$ and $GM_{Earth}=3.986004\times 10^{5}km^3s^{-2}$ (as recommended by TODO). The maximum speed of a celestial body is thus $v_{max}=\sqrt{\frac{GM}{a}}\times \sqrt{\frac{1+e}{1-e}}$. Expressed numerically, we have:

% TODO: http://maia.usno.navy.mil/NSFA/NSFA_cbe.html\#ConstGrav2009

\begin{itemize}
\item for planets, $v_{max} \approx{} \frac{364297.18}{\sqrt{a}}\times \sqrt{\frac{1+e}{1-e}}$\
\item for the Moon, $v_{max} \approx{} \frac{631.34}{\sqrt{a}}\times \sqrt{\frac{1+e}{1-e}}$
\end{itemize}

These formulas correspond to the formulas page 238 of \cite{Meeus}, except that the latter have $a$ in astronomical unit. A few calculations will require minimum speed of the plantets, for it we simply take the last formulas, replacing $e$ by $-e$.

We will also need the shortest possible distance from Earth to another celestial body. For the Sun and Moon this is straightforward; for other planets we obtain a minimal bound by substracting the sun distance of the Earth at Aphelion and of the planet at Perihelion (for outer planets, or the opposite for inner planets). This gives the following formulas:

\begin{itemize}
\item for the Sun, $r_{min}=a_{Earth}(1-e_{Earth})$
\item for the Moon, $r_{min}=a_{Moon}(1-e_{Moon})$
\item for inner planets, $r_{min}=a_{Earth}(1-e_{Earth}) - a_{p}(1+e_{p})$
\item for outer planets, $r_{min}=a_{p}(1-e_{p}) - a_{Earth}(1+e_{Earth})$
\end{itemize}

Note that these formulas are not precise as they rely simply on a very theoretical motion, For planets, we'll use the data given in \cite{NASA-factsheet} which are more precise. We can now compute the values of Table \cite{table:planetvalues}. We do the same for maximum distance $r_{max}$.

\begin{table}
\centering
\begin{tabular}{|l|S[table-format=1.2,table-figures-exponent=1]|S[table-format=1.2,table-figures-exponent=1]|S[table-format=2.2,table-figures-exponent=0]|S[table-format=2.2,table-figures-exponent=0]|}
\hline
\multicolumn{1}{|c|}{\textbf{Celestial body}} & \multicolumn{1}{c|}{\textbf{$r_{min}$} (km)} & \multicolumn{1}{c|}{\textbf{$r_{max}$} (km)} & \multicolumn{1}{c|}{\textbf{$v_{max}$} (km/s)} & \multicolumn{1}{c|}{\textbf{$v_{min}$} (km/s)} \\\hline
Sun & 1.47e8 & 1.52e8 & 30.29 & 29.29\\\hline % mean: 150km
Moon & 3.57e5 & 4.07e5 & 1.08 & 0.96\\\hline % mean: 385kkm
Mercury & 7.7e7 & 221.9 & 58.98 & 38.86\\\hline % mean: 100Mkm
Mars & 5.5e7 & 4.01e8 & 26.5 & 21.97\\\hline % mean: 225Mkm
Venus & 3.8e7 & 2.61e8 & 35.26 & 34.78\\\hline % mean: 150Mkm
Saturn & 1.2e9 & 1.66e9 & 10.18 & 9.09\\\hline % mean: 1450Mkm
Jupiter & 5.9e8 & 9.68e8 & 13.71 & 12.43\\\hline % mean: 770kMkm
\end{tabular}
\caption{Planet caracteristics used in the calculations}
\label{table:planetvalues}
\end{table}

\mysection{Thanks}

I would like to thank a lot people without who this paper would never have been possible.

First of course my masters who would certainly like to remain anonymous, for the inspiration to do things well.

Also to Edward Henning, for all the work he's been doing on the Tibetan Calendar, his very precious advices and great disponibility.

Venerable Dr. Knuth provided both many theoretical advances in the field and the most precious tools (\TeX ) that allowed to build this document.

My gratitude goes also to Mathieu Chevrier for his wide knowledge on floating point numbers.

\mysection{License}

This work is under the Creative Commons Attribution-Share Alike 3.0 Unported License. See \cite{CCASA} for details and the complete license.


\mychapter{What precision for calendrical calculations?}

\mysection{Trueness and precision}

This is quite important here to differenciate trueness and precision, both being part of accuracy. These terms are officially defined in \cite{VIM}\footnote{in accordance to ISO-5724\cite{ISO5725}} as closeness of agreement between:

\begin{description}
\item[trueness] the average of an infinite number of replicate measured quantity values and a reference quantity value
\item[precision] indications or measured quantity values obtained by replicate measurements on the same or similar objects under specified conditions
\item[accuracy] a measured quantity value and a true quantity value of a measurand
\end{description}

It makes sense to translate them to astronomical calculations, taking a calculation as a \emph{measure}. \ref{ATP} is useful to get a better understanding.

\begin{figure}[h]
\centering
\def\svgwidth{10cm}
\input{ATP.pdf_tex}
\label{ATP}
\caption{Accuracy, Precision and Trueness of calculations\footnote{This image comes from Wikipedia and is under the same license as this document}}
\end{figure}

\mysection{Precision of different computer methods}

We will take here the 

\mysection{Scales used in lunisolar calendars}

\mysubsection{The tibetan calendar}

\mysubsection{Other calendars}


\mychapter{The different calculation methods}

\mysection{Variations Séculaires des Orbites Planétaires (VSOP)}

\mysubsection{vsop87}

\mysubsection{vsop2013}

\mysection{ELP/MPP02 and LEA-406}

\mysection{JPL Ephemeris}


\mychapter{Timescales}

\mysection{Introduction}

It might first seem strange to talk about timescales, the obvious belief is that there is only one timescale and we seasure events on it. Sadly, this is not the case and some errors might occur due to this belief. In this chapter, we will briefly describe the different timescales, their relation, and their impact on calculations.

\mysubsection{Impact of the error on time}

As all our numerical data is in the second of arc unit, we need to first make some calculations that will allow conversion from an error on time (expressed in seconds) in an error of angle, expressed in seconds of arc. Of course, this conversion depends on too many things to get even an approximation of the function giving this conversion. But what's interesting for us is the maximum conversion rate, meaning the conversion rate in the case where it will induce the maximal anglar error. 

As we are interested in majoring this conversion rate, we will also use many pessimistic approximations. The data we'll use for celestial bodies correspond to the following case:

\begin{itemize}
\item the celestial bodies are at their closest position from the Earth
\item the celestial bodies are at their maximum speed (even the Earth)
\item the observation takes place at the top of Mt Everest, on the Ecliptic, when the observer is closest to the celestial body
\end{itemize}

As we consider short times, we'll make the following approximations:
\begin{itemize}
\item the planet goes in a linear direction
\item due to the rotation of the Earth on itself, the observer goes in the opposite linear direction
\item due to the rotation of the Earth around the Sun, the observer goes in the same linear direction as the one just before
\end{itemize}

A few notations:
\begin{itemize}
\item $r_{min}$ is the distance between the observer and the center of the planet, which we'll consider constant because of the short times
\item $v_{p}$ is the maximum linear speed of the planet
\item $v_{obs}=v_{E}+v_{S}$ is the maximum linear speed of the observer, the sum of the maximum linear speed of the observer due to the rotation of the Earth on itself and around the Sun
\end{itemize}

We'll express distances in kilometers, times in seconds and speeds in kilometers by second.

A first thing can be to determine $v_{obs}=v_E+v_S$. If we consider the observer as we defined it, and say Earth is round, in a 24h day, he'll travel $2\pi\times r_{E}$ where $r_{E}=R_\oplus+alt._{obs}=6375.9km$ is the distance between him and the geocenter. We have thus 

\begin{equation}
v_E = \frac{2\pi r_E}{86400} = 0.46km/s
\end{equation}

Now for $v_S$, we easily major it by $v_S<110000km/h=3.96\time 10^8km/s$. We have thus neglect $v_E$ and major $v_{obs}$:

\begin{equation}
v_{obs} < 4\time 10^8km/s
\end{equation}

Now, the maximum error angle by second is the difference between the angle and the angle one second later. It has its maximum when the first angle is 0, so we just need to calculate the angle after one second. This angle corresponds to Fig~\ref{maxangularerror}.

\begin{figure}
\centering 
\begin{tikzpicture}[scale=2]
\draw [<-] (-2,0) node[anchor=north] {$P_{obs}(t_1)$} -- node[anchor=north] {$\overrightarrow{v_{obs}}$} (0,0) node[anchor=north] {$P_{obs}(t_0)$} -- %node[northwest=2mm] {$r_{min}$} 
(0,2) node[anchor=south] {$P_{cb}(t_0)$};
\node [anchor=center] at (-0.25,1.35) {$r_{min}$};
\draw [->] (0,2) -- node[anchor=south] {$\overrightarrow{v_{cb}}$} (2,2) node[anchor=south] {$P_{cb}(t_1)$};
\draw [gray] (2,2) -- (-2,0);
\draw (-1.3,0) arc (0:26.56:0.7);
\node [anchor=center] at (-1,0.2) {$\Theta_{max}$};
\end{tikzpicture}
\caption{Maximum angular error for short times}\label{maxangularerror}
\end{figure}

This figure can be easily simplified in a simple triangle and we can deduce 

\begin{equation}
\Theta_{max}=atan(\frac{v_{obs}+v_{cb}}{r_{min}})
\end{equation}



\mysection{Astronomical timescales}

\mysubsection{Atomic time (TAI)}

\mysubsection{Earth Rotation Time}

\mysubsection{Dynamical Time}

\mysubsubsection{Based on TAI}

\mysubsubsection{Truely relativistic time}

\mysection{Time used in calendrical calculations}

\mysubsection{Mean Sun?}

\mysection{Conversions}


\mychapter{Coordinate systems}

\mychapter{Errors due to Time and Coordinate system calculations}

\mysection{Delta T uncertainty}

\mysection{Nutation Effect}

\mysection{Obliquity of the Ecliptic}

\mysection{Earth Rotation Angle (ERA)}

\mysection{Position of the observer}

Calendrical calculations are supposedly made by humans above the sea, not at the geocenter. So the observed longitude of a celestial body 

$R_o$ is the distance of the observer from the geocenter ($R_E+altitude$)
$D_{E-M}$ is the Earth-Moon distance
$\Theta$ is the resulting error

The error will be maximized by an observation of the moon when it's $90°$ from vernal equinox and at its perigee (about $D_{E-Mmin}\approx360000km$) made by someone on the top of the Everest ($R_o=R_E+8,848km$) and considering the Everest is on the Ecliptic. In this case we would have:

$$\Theta_{max} = atan(R_o/D_{E-Mmin}) = 3655as$$

% atan(6379.85/360000) = .01771995065419387521rad = 3655s

Which is an error to be avoided by all means!

If we take an observation at sea level, with a $30°$ angle and an average distance (Sun-Earth distance for inner planets and Sun-planet distance for outer ones), we can compute the following table:

\begin{center}
\sisetup{table-figures-exponent = 1}
\begin{tabular}{|l|S|S|}
\hline
\textbf{Celestial body} & \multicolumn{1}{c|}{\textbf{$\Theta_{max}$}} & \multicolumn{1}{c|}{\textbf{$\Theta$}}  \\\hline
Sun & 9 & 4\\\hline % min: 147Mkm, mean: 150km
Moon & 3655 & 1706\\\hline % min: 360kkm, mean: 385kkm
Mercury & 17 & 7\\\hline %  min: 77 Mkm, mean: 100Mkm
Mars & 24 & 3\\\hline % min: 54.6Mm, mean: 225Mkm
Venus & 35 & 4\\\hline % min: 38Mkm, mean: 150Mkm
Saturn & 1 & 453\\\hline % min  1200Mkm, mean: 1450Mkm
Jupiter & 2e-3 & 8e-4 \\\hline % min: 629kMkm, mean: 770kMkm
\end{tabular}
\end{center}

\mychapter{Other inaccuracies}

\mysection{iterations in the light tralvel time}

\mysection{Aberration of light}



\end{document}
