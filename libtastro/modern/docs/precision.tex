\documentclass[%
a4paper,% paper size
pagesize,%
12pt,% text size
parskip=off,% inter-paragraph space, automatically sets paragraph indentation
bibliography=totoc,% bibliography in the table or contents
numbers=noenddot,%
DIV=12,% margin settings
twoside=semi,% right and left difference, but same margins
headings=small,% small headings
]{scrbook}

\usepackage{eroux}

\title{Astronomical calculations in lunisolar calendars}
\subtitle{Overview of accuracy and optimization possibility}
\author{Élie Roux \href{mailto:elie.roux@telecom-bretagne.eu}{<elie.roux@telecom-bretagne.eu>}} 
\date{\today} 

\pagestyle{scrheadings}
\ifoot[]{} 
\cehead[]{\textsc{\rightmark}}
\cohead[]{\textsc{Astronomical calculations in lunisolar calendars}}
% et enfin le numéro de page dans le pied de page extérieur
\ofoot[\oldstylenums\thepage]{\oldstylenums\thepage}
\cfoot[]{}
\ohead[]{}
\ihead[]{}

\let\mychapter\chapter
\let\mysection\section
\let\mysubsection\subsection
\let\mysubsubsection\subsubsection
\let\mysubsubsubsection\subsubsubsection

\def\TODO{}

\begin{document}

\maketitle

\tableofcontents
%\cleardoubleevenemptypage
%\clearpage
\newpage

\mychapter{Introduction}

\mysection{What it this document?}

This article aims at giving an overview of the main inaccuracies encounterable in astronomical calculations when used in calendrical calculations.

\mysection{Genesis}

Working on Tibetan calendar calculations, I realized that there was no documentation on how to use the main astronomical calculations (such as VSOP87) to get the data necessary to build a calendar. These include the different approximations we would naturally make with the result, the induced discrepancy of which I have never seen documented.

I guess many people did these calculations, such as the authors of the main astronomical libraries, but no documentation is available on the global accuracy of their calculations nor on why they would take into account a phenomenon or not.

Instead of redoing these calculations myself, I decided to compute them fully and document them, in order:
\begin{itemize}
\item to make the accuracy gain of some considerations documented
\item to analyze and document the precision of the calculations used in my astronomical library\TODO
\end{itemize}

This article is made from the point of view of calendrical calculations but will be meaningful with any calculation from the point of view of a human observer on the Earth.

This article will not give a very precise average error for all phenomena, but:
\begin{itemize}
\item the maximum error
\item an error computed with what seems average values
\item a conclusion on the necessity to take the phenomenon into account according to the desired precision
\end{itemize}



\mysection{Conventions used in this article}

\mysubsection{Assertions}

In this article, we will consider that latest JPL ephemeris \TODO are fully accurate. 

We consider that all computer calculations are approximated as defined by IEEE 754/784, so calculations are supposed to happen in C99 or Fortran 2003. If you use a non-prehistoric processor and compiler, it should be the case though.

\mysubsection{Notation and vocabulary}

\mysubsubsection{Trueness and precision}

This is quite important here to differenciate trueness and precision, both being part of accuracy. These terms are officially defined in \cite{VIM} as closeness of agreement between:

\begin{description}
\item[trueness] the average of an infinite number of replicate measured quantity values and a reference quantity value
\item[precision] indications or measured quantity values obtained by replicate measurements on the same or similar objects under specified conditions
\item[accuracy] a measured quantity value and a true quantity value of a measurand
\end{description}

It makes sense to translate them to astronomical calculations, taking a calculation as a \emph{measure}. Figure~\ref{ATP} is useful to get a better understanding.

\begin{figure}[h]
\centering
\def\svgwidth{10cm}
\input{ATP.pdf_tex}
%\caption{Accuracy, Precision and Trueness of calculations\\This image comes from Wikipedia and is under the same license as this document\TODO}
\label{ATP}
\end{figure}

\mysubsection{Astronomical values}

\mysubsubsection{Orbital caracteristics}

For future calculations, we'll need a set of orbital values which we'll present and justify here. The values we'll use for $a$ and $e$, presented in Table \cite{table:planetorbitalvalues}, come from \cite{NASA-factsheet}. % (TODO: http://nssdc.gsfc.nasa.gov/planetary/planetfact.html)

\begin{table}
\centering
\begin{tabular}{|l|S[table-format=1.3,table-figures-exponent=1]|S[table-format=1.2,table-figures-exponent=1]|}
\hline
\multicolumn{1}{|c|}{\textbf{Celestial body}} & \multicolumn{1}{c|}{\textbf{$a$} (km)} & \multicolumn{1}{c|}{\textbf{$e$}} \\\hline
Sun (Earth) & 1.496e8 & 1.67e-2\\\hline
Moon & 3.84e5 & 5.49e-2\\\hline
Mercury & 5.79e7 & 2.056e-1\\\hline
Mars & 2.279e8 & 9.35e-2\\\hline
Venus & 1.082e8 & 6.77e-3\\\hline
Saturn & 1.433e9 & 5.65e-2\\\hline
Jupiter & 7.786e8 & 4.89e-2 \\\hline
\end{tabular}
\caption{Planet caracteristics used in the calculations}
\label{table:planetorbitalvalues}
\end{table}

Then we need the maximum speed of celestial bodies. The first estimation of it (which we'll use) is obtained by the Kepler laws: $$\sqrt{2GM}\times \sqrt{\frac{1}{r}-\frac{1}{2a}}$$. We use $G=6.67384\times 10^{2} km^3kg^{-1}s^{-2}$\footnote{As recommended by \cite{CODATA}}, $GM_{Sun}=1.3271244\times 10^{11} km^3s^{-2}$ and $GM_{Earth}=3.986004\times 10^{5}km^3s^{-2}$ (as recommended by TODO). The maximum speed of a celestial body is thus $v_{max}=\sqrt{\frac{GM}{a}}\times \sqrt{\frac{1+e}{1-e}}$. Expressed numerically, we have:

% TODO: http://maia.usno.navy.mil/NSFA/NSFA_cbe.html\#ConstGrav2009

\begin{itemize}
\item for planets, $v_{max} \approx{} \frac{364297.18}{\sqrt{a}}\times \sqrt{\frac{1+e}{1-e}}$\
\item for the Moon, $v_{max} \approx{} \frac{631.34}{\sqrt{a}}\times \sqrt{\frac{1+e}{1-e}}$
\end{itemize}

These formulas correspond to the formulas page 238 of \cite{Meeus}, except that the latter have $a$ in astronomical unit. A few calculations will require minimum speed of the plantets, for it we simply take the last formulas, replacing $e$ by $-e$.

We will also need the shortest possible distance from Earth to another celestial body. For the Sun and Moon this is straightforward; for other planets we obtain a minimal bound by substracting the sun distance of the Earth at Aphelion and of the planet at Perihelion (for outer planets, or the opposite for inner planets). This gives the following formulas:

\begin{itemize}
\item for the Sun, $r_{min}=a_{Earth}(1-e_{Earth})$
\item for the Moon, $r_{min}=a_{Moon}(1-e_{Moon})$
\item for inner planets, $r_{min}=a_{Earth}(1-e_{Earth}) - a_{p}(1+e_{p})$
\item for outer planets, $r_{min}=a_{p}(1-e_{p}) - a_{Earth}(1+e_{Earth})$
\end{itemize}

Note that these formulas are not precise as they rely simply on a very theoretical motion, For planets, we'll use the data given in \cite{NASA-factsheet} which are more precise. We can now compute the values of Table \cite{table:planetvalues}. We do the same for maximum distance $r_{max}$.

\begin{table}
\centering
\begin{tabular}{|l|S[table-format=1.2,table-figures-exponent=1]|S[table-format=1.2,table-figures-exponent=1]|S[table-format=2.2,table-figures-exponent=0]|S[table-format=2.2,table-figures-exponent=0]|}
\hline
\multicolumn{1}{|c|}{\textbf{Celestial body}} & \multicolumn{1}{c|}{\textbf{$r_{min}$} (km)} & \multicolumn{1}{c|}{\textbf{$r_{max}$} (km)} & \multicolumn{1}{c|}{\textbf{$v_{max}$} (km/s)} & \multicolumn{1}{c|}{\textbf{$v_{min}$} (km/s)} \\\hline
Sun & 1.47e8 & 1.52e8 & 30.29 & 29.29\\\hline % mean: 150km
Moon & 3.57e5 & 4.07e5 & 1.08 & 0.96\\\hline % mean: 385kkm
Mercury & 7.7e7 & 221.9 & 58.98 & 38.86\\\hline % mean: 100Mkm
Mars & 5.5e7 & 4.01e8 & 26.5 & 21.97\\\hline % mean: 225Mkm
Venus & 3.8e7 & 2.61e8 & 35.26 & 34.78\\\hline % mean: 150Mkm
Saturn & 1.2e9 & 1.66e9 & 10.18 & 9.09\\\hline % mean: 1450Mkm
Jupiter & 5.9e8 & 9.68e8 & 13.71 & 12.43\\\hline % mean: 770kMkm
\end{tabular}
\caption{Planet caracteristics used in the calculations}
\label{table:planetvalues}
\end{table}

\mysection{Thanks}

I would like to thank a lot people without who this paper would never have been possible.

First of course my masters who would certainly like to remain anonymous, for the inspiration to do things well.

Also to Edward Henning, for all the work he's been doing on the Tibetan Calendar, his very precious advices and great disponibility.

Venerable Dr. Knuth provided both many theoretical advances in the field and the most precious tools (\TeX ) that allowed to build this document.

My gratitude goes also to Mathieu Chevrier for his wide knowledge on floating point numbers.

\mysection{License}

This work is under the Creative Commons Attribution-Share Alike 3.0 Unported License. See \cite{CCASA} for details and the complete license.


\mychapter{What precision for calendrical calculations?}

\mysection{Trueness and precision}

This is quite important here to differenciate trueness and precision, both being part of accuracy. These terms are officially defined in \cite{VIM}\footnote{in accordance to ISO-5724\cite{ISO5725}} as closeness of agreement between:

\begin{description}
\item[trueness] the average of an infinite number of replicate measured quantity values and a reference quantity value
\item[precision] indications or measured quantity values obtained by replicate measurements on the same or similar objects under specified conditions
\item[accuracy] a measured quantity value and a true quantity value of a measurand
\end{description}

It makes sense to translate them to astronomical calculations, taking a calculation as a \emph{measure}. \ref{ATP} is useful to get a better understanding.

\begin{figure}[h]
\centering
\def\svgwidth{10cm}
\input{ATP.pdf_tex}
\label{ATP}
\caption{Accuracy, Precision and Trueness of calculations\footnote{This image comes from Wikipedia and is under the same license as this document}}
\end{figure}

\mysection{Precision of different computer methods}

We will take here the 

\mysection{Scales used in lunisolar calendars}

\mysubsection{The tibetan calendar}

\mysubsection{Other calendars}


\mychapter{Precision of different computer methods}

In the process of improving precision in calculations, it is very important to understand the limits of the underlying hardware and software, and the cost of going over them.

\mysection{Different computer representations}

\mysubsection{General representation}

Real numbers are represented by floating point numbers, that can be schematized as:
\begin{itemize}
\item a sign, encoded on 1 bit
\item an exponent $e$, encoded in several bits (11 on double precision)
\item a significand $s$, encoded in the rest of the bits (52 on double precision), these bits are called the mantissa
\end{itemize}

In arbitrary precision libraries, operations are made in software and thus doesn't depend too much on hardware representation, at the cost of an important speed loss. The system is the same though, but the size of the exponent and the mantissa is given by the user\footnote{Note that the mainstream GMP/MPFR doesn't allow to set truely arbitrary exponent}.

The representation size of a floating point number vary according to hardware architecture and user input. We will take into consideration here the most common types (which should cover 99.99\% of use cases). We describe here the IEEE-754 standard as implemented in C99, but there are strict equivalents in Fortran.

\mysubsection{Norms and definitions}

This paragraph describes the most simple description of floats, namely the description of the main types as in the IEEE-754 norm. This can be summarized in the following table:

\begin{center}
\vspace{\spacearoundtables}
\begin{tabular}{|l|S|S|S|}
\hline
\multicolumn{1}{|c|}{\textbf{name}} & \multicolumn{1}{c|}{\textbf{total}} & \multicolumn{1}{c|}{\textbf{exponent}} & \multicolumn{1}{c|}{\textbf{mantissa}} \\\hline
Single precision & 32 & 8 & 23 \\\hline
Double precision & 64 & 11 & 52 \\\hline
Quadruple precision & 128 & 15 & 112 \\\hline
\end{tabular}
\vspace{\spacearoundtables}
\end{center}

But as we will see, there are some discrepancies between this and reality, due to several implementation choices that we will detail in the next sections.

\mysubsection{Hardware implementation}

There is currently no widespread hardware that can make operations on floating points with more than 80 bits\footnote{though the z/Architecture of IBM Mainstream Servers can compute 128-bit float operations.}, and this limitation tends to be more and more restrictive. This section will explain the different hardware implementations. It is a very important topic to understand as it is non-obvious and implementation choices might seem very strange.

First we can distinguish between two floating point computing hardwares:
\begin{enumerate}
\item[FPU]\footnote{\emph{Floating Point Unit}, or x87.} This is the historical hardware introduced by Intel, present in all x86 and x86\_64 computers. This hardware has 80-bits registers and thus cannot make more precise operations.
\item[SIMD]\footnote{\emph{Single Instruction, Multiple Data}.} Single operations work with at most 64-bit registers and are thus less precise. SIMD offers thought the possibility to make several operations at the same time, hence the trend to prefer it. SSE, AVX, NEON, etc. are all following this architecture. Most complex operations such as trigonometric ones are not implemented and fall back on \emph{FPU}.
\end{enumerate}

Floating point numbers represented in 80-bit registers follow the \emph{extended precision} of IEEE-754 standards and is composed of a 15-bit exponent and a 64-bits mantissa.

Modern x86 and x86\_64 hardware have both SIMD (SSE or AVX) and FPU.

On decent compilers, it is possible to decide between SIMD, FPU or both\footnote{on \texttt{gcc}, the \texttt{-mfpmath} switch allows this, see \url{http://gcc.gnu.org/onlinedocs/gcc/i386-and-x86_002d64-Options.html} and \url{http://gcc.gnu.org/wiki/Math_Optimization_Flags}}. It is thus necessary to study this topic if you want to work on precision. By default, on x86 processors, the FPU is used, while SIMD is the default on x86\_64 processors and ARM processors.

It is important to note that on ARM processors (present on smartphones, tablets, etc.), floating point operations are sometimes done in a VFP, a kind of modified FPU. This VFP has no clear specification, but it seems not to have 80-bit registers, so smartphones and tablets should be considered to have only 64-bit registers. Also, VFP is not mandatory in the ARM architecture, and can be replaced by a software fallback (softvp)\footnote{see \url{http://infocenter.arm.com/help/index.jsp?topic=/com.arm.doc.dui0348c/CIHCJIFE.html}} so the target architecture is definitely to be inspected carefully!

Another form of harware implementation was present in PowerPC and SPARC processors, where 128-bit numbers could be encoded in two 64-bit registers with operations combining them. This arithmetic had a precision of about 106-bit.

\mysubsection{Software fallback}

It is possible to make floating point arithmetic with software, at the cost of very significant speed reduction. These implementations are of two types:

\begin{enumerate}
\item[\enumstyle{Compiler}] Sometimes languages specifications (or extensions) cannot be implemented with hardware, and thus compilers provide a software fallback for some types
\item[\enumstyle{Libraries}] Some libraries can do arbitrary precision floating point arithmetic. The most famous ones being GMP\footnote{\url{http://gmplib.org/}} and MPFR\footnote{\url{http://www.mpfr.org/}}.
\end{enumerate}

\mysubsection{C implementation}

This part will describe C99, but anyone using a language for astronomical calculations should get knowledge on this topic for the language he chooses.

The discrepancy between hardware and a naive float approach is naturally also present between the naive approach and the various implementations of compilers. Indeed, compilers tend to stick as close to hardware as possible (which is their role) in order to use as few costy software emulation as possible. 

As we've seen, almost no hardware is capable of more than 80-bit calculations, thus hardware 128-bit calculation is impossible. We can also safely assume that users using the C99 keyword \texttt{long double} don't want software emulation. To solve this dilemma, compilers chose different options, the main things to know being summed up in the next paragraphs.

\mysubsubsection{\texttt{double} on FPU}

\texttt{double} keyword always uses 64-bit representation in memory, but when it comes to representation in (80-bit) FPU registers, two behaviours are possible:
\begin{enumerate}
\item[\enumstyle{80-bit mode}] calculations and intermediate values are made in 80-bit, which improves the precision of the calculations\footnote{but may lead in wrong comparaison results, see \url{http://gcc.gnu.org/wiki/x87note}}
\item[\enumstyle{64-bit mode}] the mantissa of intermediate values is rounded to 53 bits (thus giving double precision precision)\footnote{This is what the \texttt{-mpc64} option of gcc or the \texttt{/fp:precise} option of MSVC do, see \url{http://gcc.gnu.org/onlinedocs/gcc/i386-and-x86\_002d64-Options.html} and \url{http://msdn.microsoft.com/en-us/library/e7s85ffb.aspx}.}.
\end{enumerate}

The FPU can also have a 32-bit mode based on the same principles as the 64-bit mode. Compilers Macros allow to change FPU mode during runtime. This is the case of \texttt{\_controlfp} of MSVC\footnote{\url{http://msdn.microsoft.com/en-us/library/e9b52ceh\%28VS.71\%29.aspx}} and \texttt{\ _FPU\_SETCW} of gcc{see \texttt{man 3 \_\_setupcw}}.

\texttt{double} on ARM has a 64 bit representation and is always computed on VFP. It's important to note here that the NEON instructions don't provide double precision operations\footnote{see \url{http://www.arm.com/products/processors/technologies/neon.php}}.

\mysubsubsection{\texttt{long double} on FPU}

\texttt{long double} can have several representations:
\begin{enumerate}
\item[\enumstyle{64-bit}] this is the default on MSVC, \texttt{long double} being a synonym of \texttt{double}\footnote{see \url{http://msdn.microsoft.com/en-us/library/9cx8xs15.aspx}}
\item[\enumstyle{96-bit}] this is the default on gcc. In this case, 16 bits are not used in calculations in 80-bit mode.
\item[\enumstyle{128-bit}] this is the case on gcc with the \texttt{-m128bit-long-double}\footnote{see \url{http://gcc.gnu.org/onlinedocs/gcc/i386-and-x86\_002d64-Options.html}}. 48 bits are not used in calculations.
\end{enumerate}

For the sake of completeness, it's important to notice here that gcc allows \texttt{\_\_float80}\footnote{\url{http://gcc.gnu.org/onlinedocs/gcc/Floating-Types.html}} type which is a synonym of \texttt{long double} on x86 and x86\_64 architectures.

\mysubsubsection{\texttt{long double} on SIMD and ARM}

The x86\_64 ABI\footnote{\url{http://www.x86-64.org/documentation/abi.pdf}} states that \texttt{long double} has intermediate values and calculations is in extended precision (80-bit), all operations being performed on the FPU; so no SIMD code will be used with the keyword \texttt{long double}.

On ARM, \texttt{long double} is a synonym of double\footnote{see \url{http://infocenter.arm.com/help/index.jsp?topic=/com.arm.doc.dui0067d/BABFCGFC.html}}.

\mysubsubsection{Summing up}

We can thus construct the following table summing up the previous paragraphs, giving some name conventions we'll use later. These names are just convenient and (though largely the same) not identical to IEEE-754 standard. The number of bits cells are in the form $x$/$y$, where $x$ is the number of bits used in memory and $y$ the number of bits used in calculations and intermediate values. When a "($\times x$)" is present, $x$ indicates the number of simultaneous operations on the type the architecture is capable of.

\begin{table}[h]
\begin{tabu}{|X[l]|X[c]|X[c]|X[c]|}
\hline
\rowfont[c]{\bfseries} Architecture / Compiler & float & double & long double
\\\hline
\textbf{SSE} & 32/32 ($\times$4) & 64/64 ($\times$2) & 64/64 ($\times$2)\\\hline
\textbf{AVX} & 32/32 ($\times$8) & 64/64 ($\times$4) & 64/64 ($\times$4)\\\hline
\textbf{NEON} & 32/32 ($\times$2) & unavailable & unavailable\\\hline
\textbf{ARM VFP} & 32/32 & 64/64 & 64/64\\\hline
\textbf{x86 FPU (gcc)} & 32/80 & 64/80 & 96/80\\\hline
\textbf{x86\_64 FPU (gcc)} & 32/80 & 64/80 & 128/80\\\hline
\textbf{x86 or x86\_64 FPU (msvc)} & 32/80 & 64/80 & 64/80\\\hline
\end{tabu}
\caption{C floating point types representation on different architectures}
\end{table}

\mysection{Common errors due to floating point representations}

This section is an overview of the most common errors due to floating point arithmetic, and of their solution.

\mysubsection{Introduction}

One of the problems of floating point arithmetic is that global formulas are almost inexistant and error for each floating point number manipulation should be calculated by hand, depending on the variable maxima and minima, the chosen float representation, etc.

This section will thus describe only general errors and things to know about floating point manipulation. A good introduction to this topic is \cite{Goldberg}, and \cite{Higham} a more recent and complete book. This section describes the general principles described in these.

\mysubsubsection{Notation}

We will use here analytical notation of floating points numbers we can find very commonly. We will represent a floating point number as being in the form $$d.dd...dd\times\beta^e$$where $d.dd...dd$ is the significand and has p digits, $\beta$ is the base (assumed to be even) and $e$ the exponent.

To make the link with computer representations, we would have:
\begin{itemize}
\item $\beta=2$
\item $p$ equal to the number of bits in the mantissa
\item $d.dd...dd$ the mantissa, with d in base 2, for example 1.100110011001100
\item $e$ the exponent
\end{itemize}

\mysubsubsection{Non-} % communativité

\mysubsection{Polynomial calculations}

When evaluating polynomial expressions, which are numerous in astronomical calculations, the standard method is to use Horner's method. A very simple example will show why:

If you evaluate $$3x^3 + 2x^2 + 4x + 5$$ in this very form, 10 operations will be performed (7 multiplications and 3 sums). But if you evaluate it as $$((3x + 2)x + 4)x + 5$$, only 6 operations (3 multiplications and 3 sums) are needed.

This operation reducing implies a gain in performance and also naturally in precision: the less operations, the less errors!

For further reading and error bouding, see setion 5.1 of \cite{Higham}.

\mysubsection{Fused Multiply-Add}

\mysubsubsection{Principles}

Fused Multiply-Add (FMA) is a hardware instruction that computes a multiplication and an addition. The interesting thing is that IEEE 754-2008 standard\footnote{see section 5.4.1 of \cite{IEEE754}} specifies that the whole operation should be performed as if no rounding occured, rounding only once at the end. This means that the whole operation is precise to $\frac{1}{2}\,ula$, where a multiplication then a sum would cumulate roundings and thus be precise only to $1\,ula$. There is thus, like with Horner's method, a gain in both performance and precision.

\mysubsubsection{Real-life use}

FMA are sadly not really widespread. Here is what seems to be the current situation:
\begin{itemize}
\item x87 doesn't handle it\footnote{except for a patented extension (patent US7499962, see \href{http://www.google.com/patents/US7499962}), which no common hardware seems to implement.}
\item AVX have FPA as an extension, thus only recent Intel hardware (as of end of 2013) will be able to use it\footnote{see \url{http://www.intel.com/content/www/us/en/processors/architectures-software-developer-manuals.html}\TODO}
\item ARM's VFP and NEON seem to handle FPA\footnote{see\url{http://infocenter.arm.com/help/index.jsp?topic=/com.arm.doc.dui0056d/Bcfhgcgd.html}}
\end{itemize}

The FMA had two mutually incompatible implementations, one on three operands (FMA3) and one on four operands (FMA4), each having made its way in AMD and Intel processors at different times\footnote{The only source I found for this is \url{http://en.wikipedia.org/wiki/FMA\_instruction\_set\#History}}. Thus FMA instructions will be computed by gcc only if the targer architecture is specified (or the specific \texttt{-mfma} and \texttt{-mfma4} switches). It is thus hard to make it generic\footnote{Though the very exciting gcc feature of Function Multiversionning might help a lot, see \url{http://gcc.gnu.org/onlinedocs/gcc/Function-Multiversioning.html}.}.

% http://shemesh.larc.nasa.gov/NFM2010/papers/nfm2010_14_23.pdf
% http://hal.inria.fr/docs/00/77/25/08/PDF/main.pdf
% http://frama-c.com/

Except on very specific hardwares, it is thus sadly impossible to get FPA on extended double precision. If you manage to use it, it's important to note that using FPA requires extra care about operations order and compiler optimization (see 2.6 in \cite{Higham}).

\mysubsection{Summations}

Several optimizations for sums are possible, we'll describe the most simples. An excellent reading on the topic is chapter 4 of \cite{Higham}. This topic is quite important, as astronomical calculations often sum large amounts of numbers (more than a thousand).

\mysubsubsection{Recursive summation}

Before seing how to optimize precision, let's present the precision of what is often called \emph{recursive summation}, meaning the naive approach of summing each term consecutively.

If we call $n$ the number of terms in the summation, the bound of error of recursive summation is $n\times\epsilon$ where $\epsilon$ is the error bound of one summation, $0.5\,ulp$ with IEEE-754 compliant systems. This means $n\times0.5\times2^{e-1023} = n\times2^{e-1022}$.

To take a realistic example in astronomical calculations, let's take $n=1024=2^{10}$ and let's say we're working on 64bit registers, with values ranging around $0.5$, meaning $e\approx1022$, we have thus an error bounded by $1$, meaning totally unreliable results!

Of course this is a bound which will most certainly never be reached, but it's definitely something to pay attention to.

A refinement of this value is given in Table 4.1 of  \cite{Higham}.

\mysubsubsection{Ascending order}

It is a good practice, for sums of more than two floating point numbers, to sum in ascending order of exponent (if we can know or guess it), treating small numbers before big ones. The reason for this is fairly simple and can be explained by the following example.

Suppose $\beta=10$, $p=3$, $e=4$ (which means we manipule decimal numbers with three digits). Let's add $0.4$, $0.4$ and $100$. Without ascending order, we would have:

$$1.00\times 10^2 + 4.00\times 10^{-1} = 1.00\times 10^2, 1.00\times 10^2 + 4.00\times 10^{-1} = 1.00\times 10^2$$

And thus a result of 100, due to the fact that 1.004 would require a minimum value of $4$ for $p$. Now let's do it in the ascending order:

$$4.00\times 10^{-1} + 4.00\times 10^{-1} = 8.00\times 10^{-1}, 1.00\times 10^2 + 8.00 \times 10^{-1} = 1.01\times 10^2$$

which is much closer. A counter-example is if, in the same system, if you add $100$, $-100$ and $1$ in ascending order, you have:

$$1.00\times 10^{-1} + 1.00\times 10^{2} = 1.00\times 10^{2} - 1.00\times 10^{2} = 0$$

whereas in the descending order:

$$1.00\times 10^{2} - 1.00\times 10^{2} = 0.00\times 10^{0} + 1.00\times 10^{-1} = 1.00\times 10^{-1}$$

which is the true answer.

Descending order is more often more accurate though, and it seems reasonable to use it, especially since it's free if you can guess the exponent order (which is the case in most astronomical calculations).

The gain in accuracy is not very good and not easily predictable though, so this method is rarely documented as a worthy improvement. The mean error table of \cite{Higham} gives an improvement by 3 compared to standard recursive summation.

\mysubsubsection{Paired Summation}

A simple method to really improve the accuracy of calculations is to compute the sum by pairs. For example, the paired computation of $$y=x_1+x_2+x_3+x_4+x_5+x_6$$ is $$y=(x_1+x_2)+(x_3+x_4)+(x_5+x_6)$$

This very simple and straightforward improvement that has an error of $log_2(n)\times\epsilon$ instead of $n\times\epsilon$.

\mysubsubsection{Compensated Summation}\label{compensated}

The method described here was invented in the 60s, and is often known as the Kahan method, from the name of its author, William Kahan. This method gives a value of the summation which is exact by $1\,ulp$, by compensating the error each time. This is a very large improvement and allows the code to be proven.

The method used won't be detailed here as it would take too much space, but it's well documented on the Web\footnote{See for instance \url{http://en.wikipedia.org/wiki/Kahan_summation_algorithm} and \url{http://www.astro.umd.edu/~dcr/Courses/ASTR615/ps2sol/node1.html}}.

The cost is quite big though, the number of operations being multiplied by 4. Results with this method are really noticeable.

For virtually infinite precision accuracy, a "double compensated summation" exists (see Algorithm 4.3 in \cite{Higham}) taking 10 operations per additions, and needing numbers to be sorted before. This method is thus extremely slow.

\mysubsubsection{Other efficient exact sums}

New algorithms have appeared recently to compute long summations exactly, with an efficiency improved a lot. Documentations and implementations of these algorithms are not avaible freely, but an overview can be found in \cite{Langlois}\TODO . % TODO: Langlois http://hal.archives-ouvertes.fr/docs/00/73/76/17/PDF/hal-scico12.pdf

\mysubsubsection{Conclusion}

It's important to define the balance between accuracy and speed, as the range is large between very slow and very accurate functions and fast and inaccurate ones.

Something very important and documented in all books on the topic is to take into account the possibility of increasing precision of the floating point representation. Indeed, compared to single precision, a recursive summation in double precision has only small speed decrease (around $1.5times$) and better accuracy than compensated summation.

\mysubsection{Errors in trigonometrical functions}

Transcendental functions (like sin and cos) guaranteed accuracy cannot be as good as other functions due to the tablemaker's dillema. Section 8.3.10 of \cite{Intel}\TODO specifies that the error cannot exceed 1\,ulp if rounding mode is \emph{nearest even}.

% TODO: Intel:  http://www.intel.com/content/www/us/en/processors/architectures-software-developer-manuals.html

It's important to also notice that, on FPU, the Internal $\pi$ representation used in trigonometric functions has a higher precision than any possible floating point hardware representation (it has a 66bit mantissa\footnote{See \cite[Intel], 8.3.8.}). Thus moduling an angle by $2\pi$ to put it in the range will be less precise than letting the FPU do by itself.

\mysubsection{Errors induced by compilers}

Something to be very cautious about is compiler optimizations. While these are necessary to improve speed, they can sometimes break precision.

\mysubsubsection{Constants computation}

On some compilers, litteral constant evaluation (for example \texttt{cos(1.65)}) was made with the precision of the constant, and thus could hold errors. This was fixed in gcc in 4.3 version which now uses arbitrary precision\footnote{see \url{http://gcc.gnu.org/gcc-4.3/changes.html\#mpfropts}}. It is thus important to check your compiler before doing any such thing; and the best way to get portable code is to write already evaluated.

\mysubsubsection{Fast maths}

Most compilers can compile code that will run faster at the cost of precision loss. These optimizations make computations not follow the IEEE-754 standard strictly and thus give unpredictable precision. This is the case of the \texttt{-Ofast}, \texttt{-ffast-math} and \texttt{-funsafe-math-optimizations} options of gcc. These would be useful in case of speed optimization.

\mysubsubsection{Arithmetic reduction}

The most dangerous compiler optimization is arithmetic reduction, as it might break precision optimizations. The principle is that compilers will try to treat operations as operations on real numbers, and thus consider them distributive and associative. At a certain optimization level, it will even totally kill compensated summation (see \ref{compensated})\footnote{by a process described in \url{http://en.wikipedia.org/wiki/Kahan_summation_algorithm\#Computer_languages} and easily noticeable on a test.}.

Decent compilers allow to to disable optimization for one function though\footnote{for gcc, see \texttt{optimize} function attribute and \texttt{pragma GCC optimize}, \url{http://gcc.gnu.org/onlinedocs/gcc/Function-Attributes.html} and \url{http://gcc.gnu.org/onlinedocs/gcc/Function-Specific-Option-Pragmas.html\#Function-Specific-Option-Pragmas}}.

\mysection{Errors induced by floating point in common astronomical calculations}

\mysubsection{Converting floating point error in angle precision error}

\mysubsection{Errors in common astronomical calculations}

\mysubsubsection{Theorical bounds}

\mysubsubsection{Some measures}



% TODO: vector code less precise? http://software.intel.com/sites/default/files/article/326703/floating-point-differences-sept11_0.pdf p.2
% http://lipforge.ens-lyon.fr/www/crlibm/

% http://gcc.gnu.org/gcc-4.3/changes.html#mpfropts << what is this? OK: for constants, evaluated at compile-time << constant folding


\mychapter{The different calculation methods}

\mysection{Variations Séculaires des Orbites Planétaires (VSOP)}

\mysubsection{vsop87}

\mysubsection{vsop2013}

\mysection{ELP/MPP02 and LEA-406}

\mysection{JPL Ephemeris}


\mychapter{Timescales}

\mysection{Introduction}

It might first seem strange to talk about timescales, the obvious belief is that there is only one timescale and we seasure events on it. Sadly, this is not the case and some errors might occur due to this belief. In this chapter, we will briefly describe the different timescales, their relation, and their impact on calculations.

\mysubsection{Impact of the error on time}

As all our numerical data is in the second of arc unit, we need to first make some calculations that will allow conversion from an error on time (expressed in seconds) in an error of angle, expressed in seconds of arc. Of course, this conversion depends on too many things to get even an approximation of the function giving this conversion. But what's interesting for us is the maximum conversion rate, meaning the conversion rate in the case where it will induce the maximal anglar error. 

As we are interested in majoring this conversion rate, we will also use many pessimistic approximations. The data we'll use for celestial bodies correspond to the following case:

\begin{itemize}
\item the celestial bodies are at their closest position from the Earth
\item the celestial bodies are at their maximum speed (even the Earth)
\item the observation takes place at the top of Mt Everest, on the Ecliptic, when the observer is closest to the celestial body
\end{itemize}

As we consider short times, we'll make the following approximations:
\begin{itemize}
\item the planet goes in a linear direction
\item due to the rotation of the Earth on itself, the observer goes in the opposite linear direction
\item due to the rotation of the Earth around the Sun, the observer goes in the same linear direction as the one just before
\end{itemize}

A few notations:
\begin{itemize}
\item $r_{min}$ is the distance between the observer and the center of the planet, which we'll consider constant because of the short times
\item $v_{p}$ is the maximum linear speed of the planet
\item $v_{obs}=v_{E}+v_{S}$ is the maximum linear speed of the observer, the sum of the maximum linear speed of the observer due to the rotation of the Earth on itself and around the Sun
\end{itemize}

We'll express distances in kilometers, times in seconds and speeds in kilometers by second.

A first thing can be to determine $v_{obs}=v_E+v_S$. If we consider the observer as we defined it, and say Earth is round, in a 24h day, he'll travel $2\pi\times r_{E}$ where $r_{E}=R_\oplus+alt._{obs}=6375.9km$ is the distance between him and the geocenter. We have thus 

\begin{equation}
v_E = \frac{2\pi r_E}{86400} = 0.46km/s
\end{equation}

Now for $v_S$, we easily major it by $v_S<110000km/h=3.96\time 10^8km/s$. We have thus neglect $v_E$ and major $v_{obs}$:

\begin{equation}
v_{obs} < 4\time 10^8km/s
\end{equation}

Now, the maximum error angle by second is the difference between the angle and the angle one second later. It has its maximum when the first angle is 0, so we just need to calculate the angle after one second. This angle corresponds to Fig~\ref{maxangularerror}.

\begin{figure}
\centering 
\begin{tikzpicture}[scale=2]
\draw [<-] (-2,0) node[anchor=north] {$P_{obs}(t_1)$} -- node[anchor=north] {$\overrightarrow{v_{obs}}$} (0,0) node[anchor=north] {$P_{obs}(t_0)$} -- %node[northwest=2mm] {$r_{min}$} 
(0,2) node[anchor=south] {$P_{cb}(t_0)$};
\node [anchor=center] at (-0.25,1.35) {$r_{min}$};
\draw [->] (0,2) -- node[anchor=south] {$\overrightarrow{v_{cb}}$} (2,2) node[anchor=south] {$P_{cb}(t_1)$};
\draw [gray] (2,2) -- (-2,0);
\draw (-1.3,0) arc (0:26.56:0.7);
\node [anchor=center] at (-1,0.2) {$\Theta_{max}$};
\end{tikzpicture}
\caption{Maximum angular error for short times}\label{maxangularerror}
\end{figure}

This figure can be easily simplified in a simple triangle and we can deduce 

\begin{equation}
\Theta_{max}=atan(\frac{v_{obs}+v_{cb}}{r_{min}})
\end{equation}



\mysection{Astronomical timescales}

\mysubsection{Atomic time (TAI)}

\mysubsection{Earth Rotation Time}

\mysubsection{Dynamical Time}

\mysubsubsection{Based on TAI}

\mysubsubsection{Truely relativistic time}

\mysection{Time used in calendrical calculations}

\mysubsection{Mean Sun?}

\mysection{Conversions}


\mychapter{Coordinate systems}

\mychapter{Errors due to Time and Coordinate system calculations}

\mysection{Delta T uncertainty}

\mysection{Nutation Effect}

\mysection{Obliquity of the Ecliptic}

\mysection{Earth Rotation Angle (ERA)}

\mysection{Position of the observer}

Calendrical calculations are supposedly made by humans above the sea, not at the geocenter. So the observed longitude of a celestial body 

$R_o$ is the distance of the observer from the geocenter ($R_E+altitude$)
$D_{E-M}$ is the Earth-Moon distance
$\Theta$ is the resulting error

The error will be maximized by an observation of the moon when it's $90°$ from vernal equinox and at its perigee (about $D_{E-Mmin}\approx360000km$) made by someone on the top of the Everest ($R_o=R_E+8,848km$) and considering the Everest is on the Ecliptic. In this case we would have:

$$\Theta_{max} = atan(R_o/D_{E-Mmin}) = 3655as$$

% atan(6379.85/360000) = .01771995065419387521rad = 3655s

Which is an error to be avoided by all means!

If we take an observation at sea level, with a $30°$ angle and an average distance (Sun-Earth distance for inner planets and Sun-planet distance for outer ones), we can compute the following table:

\begin{center}
\sisetup{table-figures-exponent = 1}
\begin{tabular}{|l|S|S|}
\hline
\textbf{Celestial body} & \multicolumn{1}{c|}{\textbf{$\Theta_{max}$}} & \multicolumn{1}{c|}{\textbf{$\Theta$}}  \\\hline
Sun & 9 & 4\\\hline % min: 147Mkm, mean: 150km
Moon & 3655 & 1706\\\hline % min: 360kkm, mean: 385kkm
Mercury & 17 & 7\\\hline %  min: 77 Mkm, mean: 100Mkm
Mars & 24 & 3\\\hline % min: 54.6Mm, mean: 225Mkm
Venus & 35 & 4\\\hline % min: 38Mkm, mean: 150Mkm
Saturn & 1 & 453\\\hline % min  1200Mkm, mean: 1450Mkm
Jupiter & 2e-3 & 8e-4 \\\hline % min: 629kMkm, mean: 770kMkm
\end{tabular}
\end{center}

\mychapter{Other inaccuracies}

\mysection{iterations in the light tralvel time}

\mysection{Aberration of light}



\end{document}
